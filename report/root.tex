\documentclass[conference]{ieeeconf}

% Necesarry to run this code
% sudo apt-get install texlive-publishers

% The following packages can be found on http:\\www.ctan.org
%\usepackage{graphics} % for pdf, bitmapped graphics files
%\usepackage{epsfig} % for postscript graphics files
%\usepackage{mathptmx} % assumes new font selection scheme installed
%\usepackage{times} % assumes new font selection scheme installed
%\usepackage{amsmath} % assumes amsmath package installed
%\usepackage{amssymb}  % assumes amsmath package installed

\title{\LARGE \bf
   Breast Cancer Disease-Specific Survival Prediction Utilizing Semi-Supervised
   and Deep Learning Approaches
}



\author{\authorblockN{Allen Du\authorrefmark{1}, Angel Hsia\authorrefmark{2}, Misael Alanis\authorrefmark{3}, Danniv Arnon \authorrefmark{4} and Jacob Chen\authorrefmark{5}}
\authorblockA{\authorrefmark{1}Graduate Institute of Biomedical Electronics and Bioinformatics\\
National Taiwan University}
\authorblockA{\authorrefmark{2}Graduate Institute of Communication Engineering\\
National Taiwan University}
\authorblockA{\authorrefmark{3}International College\\
National Taiwan University}
\authorblockA{\authorrefmark{4}Computer Science \\
Technical University of Darmstadt}
\authorblockA{\authorrefmark{5}Department of Computer Science\\
University of Mannheim}
}

\begin{document}



\maketitle
\thispagestyle{empty}
\pagestyle{empty}


%%%%%%%%%%%%%%%%%%%%%%%%%%%%%%%%%%%%%%%%%%%%%%%%%%%%%%%%%%%%%%%%%%%%%%%%%%%%%%%%
\begin{abstract}

This is the abstract. Example text Example text Example text Example text Example text.

{\textbf{\textit{Clinical Relevance}}}\textemdash This might be clinically relevant I'm not sure, to be honest.

\end{abstract}


%%%%%%%%%%%%%%%%%%%%%%%%%%%%%%%%%%%%%%%%%%%%%%%%%%%%%%%%%%%%%%%%%%%%%%%%%%%%%%%%

\section{Introduction}

Breast cancer prognosis plays a critical role in personalized treatment planning.
This study explores prediction techniques by leveraging a dataset comprising 20
genetic expression variables, 10 clinical features, 500+ labelled rows with survival
times, and 1000 unlabelled rows. We applied semi-supervised learning (SSL) methods
to utilize the unlabelled data effectively. A Cox proportional hazards model combined
with a feed-forward neural network (DeepSurv) was used to predict survival time points,
diverging from the traditional 5-year disease-specific survival (DSS) binary
classification. Additionally, a transformer model was implemented to compare
prediction accuracy. While our approach achieved higher accuracy than the original
study, it did not surpass the original paper's AUC. Our key innovations include
the SSL methodology and the prediction of survival time points, offering a nuanced
perspective on breast cancer prognosis.

\section{Related works}

\subsection{VIME}

VIME (Variational Information Maximizing Exploration) \cite{bahri2022scarf}
is a method proposed by 
Yoon et al. in 2020. It is designed for semi-supervised learning, where the goal
is to effectively utilize both labeled and unlabeled data. VIME leverages a
variational information maximization framework to enhance the representation 
learning process. The key components of VIME include a feature encoder, a 
feature decoder, and a discriminator. The feature encoder maps input data to 
a latent space, while the feature decoder reconstructs the input data from the 
latent representation. The discriminator is used to distinguish between real
and generated samples, promoting the learning of meaningful representations.
VIME has shown promising results in various benchmark datasets, demonstrating
its effectiveness in improving the performance of semi-supervised learning tasks.

\subsection{Scarf}

SCARF (Semi-supervised Classification by Augmented Random Forest) is a semi-supervised learning method proposed by
Zhu et al. in 2019. SCARF is designed to leverage both labeled and unlabeled data
to improve the performance of classification tasks. The key idea behind SCARF is
to augment the training data with pseudo-labels generated from the unlabeled data.
SCARF utilizes a random forest classifier to assign pseudo-labels to the unlabeled
data, which are then used to train a semi-supervised classifier. The augmented
training data enables the classifier to learn from both labeled and unlabeled data,
leading to improved performance. SCARF has been shown to outperform traditional
supervised and semi-supervised learning methods in various benchmark datasets,
highlighting its effectiveness in leveraging unlabeled data for classification tasks.

\subsection{DeepSurv}

\subsection{Transformer (reference 4)}


\subsection{Other Related Works}

There are other papers that are related to this work. For example, there's this paper...



Another paper is

\subsection{Related Paper 3}

And there's this paper...



\section{METHOD}

The methods

\subsection{Models}

\begin{itemize}

\item DeepSurv
\item Transformer

\end{itemize}

Furthermore, 


\section{EXPERIMENT SETUPS AND DATASETS}

This section

\section{RESULTS}

Results are not that good :(

\subsection{Figures and Tables}

\begin{table}[h]
\caption{Model performance}
\label{table_example}
\begin{center}
\begin{tabular}{|c|c|c|}
\hline
Model & Accuracy\\
\hline
DNN & 0.73\\
\hline
\end{tabular}
\end{center}
\end{table}


   \begin{figure}[thpb]
      \centering
      \framebox{\parbox{3in}{Maybe some figure
}}
      %\includegraphics[scale=1.0]{figurefile}
      \caption{Very cool figure}
      \label{figurelabel}
   \end{figure}
   

Figures are cool

\section{DISCUSSION}

Discussion of results 

\section{CONCLUSION}

\addtolength{\textheight}{-12cm}   % This command serves to balance the column lengths
                                  % on the last page of the document manually. It shortens
                                  % the textheight of the last page by a suitable amount.
                                  % This command does not take effect until the next page
                                  % so it should come on the page before the last. Make
                                  % sure that you do not shorten the textheight too much.

%%%%%%%%%%%%%%%%%%%%%%%%%%%%%%%%%%%%%%%%%%%%%%%%%%%%%%%%%%%%%%%%%%%%%%%%%%%%%%%%



%%%%%%%%%%%%%%%%%%%%%%%%%%%%%%%%%%%%%%%%%%%%%%%%%%%%%%%%%%%%%%%%%%%%%%%%%%%%%%%%



%%%%%%%%%%%%%%%%%%%%%%%%%%%%%%%%%%%%%%%%%%%%%%%%%%%%%%%%%%%%%%%%%%%%%%%%%%%%%%%%



\bibliographystyle{IEEEtran}
\bibliography{IEEEabrv,refs}


\end{document}