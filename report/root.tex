% !TEX root = ./root.tex
%%%%%%%%%%%%%%%%%%%%%%%%%%%%%%%%%%%%%%%%%%%%%%%%%%%%%%%%%%%%%%%%%%%%%%%%%%%%%%%%
%2345678901234567890123456789012345678901234567890123456789012345678901234567890
%        1         2         3         4         5         6         7         8

\documentclass[letterpaper, 10 pt, conference]{ieeeconf}  % Comment this line out if you need a4paper

%\documentclass[a4paper, 10pt, conference]{ieeeconf}      % Use this line for a4 paper

\IEEEoverridecommandlockouts                              % This command is only needed if 
                                                          % you want to use the \thanks command

\overrideIEEEmargins                                      % Needed to meet printer requirements.

%In case you encounter the following error:
%Error 1010 The PDF file may be corrupt (unable to open PDF file) OR
%Error 1000 An error occurred while parsing a contents stream. Unable to analyze the PDF file.
%This is a known problem with pdfLaTeX conversion filter. The file cannot be opened with acrobat reader
%Please use one of the alternatives below to circumvent this error by uncommenting one or the other
%\pdfobjcompresslevel=0
%\pdfminorversion=4

% See the \addtolength command later in the file to balance the column lengths
% on the last page of the document

% The following packages can be found on http:\\www.ctan.org
%\usepackage{graphics} % for pdf, bitmapped graphics files
%\usepackage{epsfig} % for postscript graphics files
%\usepackage{mathptmx} % assumes new font selection scheme installed
%\usepackage{times} % assumes new font selection scheme installed
%\usepackage{amsmath} % assumes amsmath package installed
%\usepackage{amssymb}  % assumes amsmath package installed

\title{\LARGE \bf
Breast Cancer Disease-Specific Survival Prediction Utilizing Semi-Supervised and Deep Learning Approaches
}


\author{Allen Du,
Angel Hsia, 
Misael Alanis, 
Danniv Arnon, and
Jacob Chen% <-this % stops a space
}

\begin{document}



\maketitle
\thispagestyle{empty}
\pagestyle{empty}


%%%%%%%%%%%%%%%%%%%%%%%%%%%%%%%%%%%%%%%%%%%%%%%%%%%%%%%%%%%%%%%%%%%%%%%%%%%%%%%%
\begin{abstract}

This is the abstract. Example text Example text Example text Example text Example text.

{\textbf{\textit{Clinical Relevance}}}\textemdash This might be clinically relevant I'm not sure, to be honest.

\end{abstract}


%%%%%%%%%%%%%%%%%%%%%%%%%%%%%%%%%%%%%%%%%%%%%%%%%%%%%%%%%%%%%%%%%%%%%%%%%%%%%%%%
\section{INTRODUCTION}

Breast cancer is kind of a big issue I think.

\section{RELATED WORKS}

\subsection{VIME (reference 1)}

First, Example text Example text Example text Example text.

\subsection{SCARF (reference 2)}

Another paper is

\subsection{Related Paper 3}

And there's this paper...



\section{METHOD}

The methods

\subsection{Models}

\begin{itemize}

\item DeepSurv
\item Transformer

\end{itemize}

Furthermore, 


\section{EXPERIMENT SETUPS AND DATASETS}

This section

\section{RESULTS}

Results are not that good :(

\subsection{Figures and Tables}

\begin{table}[h]
\caption{Model performance}
\label{table_example}
\begin{center}
\begin{tabular}{|c|c|c|}
\hline
Model & Accuracy\\
\hline
DNN & 0.73\\
\hline
\end{tabular}
\end{center}
\end{table}


   \begin{figure}[thpb]
      \centering
      \framebox{\parbox{3in}{Maybe some figure
}}
      %\includegraphics[scale=1.0]{figurefile}
      \caption{Very cool figure}
      \label{figurelabel}
   \end{figure}
   

Figures are cool

\section{DISCUSSION}

Discussion of results 

\section{CONCLUSION}

\addtolength{\textheight}{-12cm}   % This command serves to balance the column lengths
                                  % on the last page of the document manually. It shortens
                                  % the textheight of the last page by a suitable amount.
                                  % This command does not take effect until the next page
                                  % so it should come on the page before the last. Make
                                  % sure that you do not shorten the textheight too much.

%%%%%%%%%%%%%%%%%%%%%%%%%%%%%%%%%%%%%%%%%%%%%%%%%%%%%%%%%%%%%%%%%%%%%%%%%%%%%%%%



%%%%%%%%%%%%%%%%%%%%%%%%%%%%%%%%%%%%%%%%%%%%%%%%%%%%%%%%%%%%%%%%%%%%%%%%%%%%%%%%



%%%%%%%%%%%%%%%%%%%%%%%%%%%%%%%%%%%%%%%%%%%%%%%%%%%%%%%%%%%%%%%%%%%%%%%%%%%%%%%%



\begin{thebibliography}{99}

\bibitem{c1} First Reference 

\end{thebibliography}


\end{document}